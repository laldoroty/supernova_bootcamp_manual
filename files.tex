\section{Working with Files}
\subsection{Rotating an image and updating the fits header manually}

In your headers, there's something called the \textit{``CD matrix''}, which describes both the rotation and pixel scale of your image. You may have \texttt{CDELT1} and \texttt{CDELT2} (pixel scale, x and y axis, respectively), as well as \texttt{CROTA2} or \texttt{ORIENTAT} (rotation angle), instead. There is no \texttt{CROTA1}. I didn't make the rules, okay? The relationship between these values is:
\begin{equation}
    \begin{bmatrix}
    \texttt{CD1\_1} & \texttt{CD1\_2} \\
    \texttt{CD2\_1} & \texttt{CD2\_2}
    \end{bmatrix}
    = 
    \begin{bmatrix}
    \texttt{CDELT1}\cos{(\texttt{CROTA2})} & -\texttt{CDELT2}\sin{(\texttt{CROTA2})} \\
    \texttt{CDELT1}\sin{(\texttt{CROTA2})} & \texttt{CDELT2}\cos{(\texttt{CROTA2})}
    \end{bmatrix}
\end{equation}

You may recognize this as being very similar to a rotation matrix... that's because it is! \\

If you already have your matrix in terms of \texttt{CD1\_1} etc. instead of \texttt{CDELT} and \texttt{CROTA}, then you can ignore the above. \\

First, figure out what angle you want to rotate your image through. You can pick the angle in the \texttt{ORIENTAT/CROTA2} keyword to rotate it back to 0 degrees. Then, build your rotation matrix. You'll use this for updating the image header. To rotate the image,  you will use \texttt{scipy.ndimage.rotate}.

\begin{minted}[
    bgcolor=lightgray,
    frame=leftline,
    framesep=-3mm]
    {python}

    import numpy as np
    from scipy.ndimage import rotate
    from astropy.io import fits

    hdu = fits.open('/path/to/data.fits')
    header = hdu[0].header
    img = hdu[1].data
    
    CDmat = np.array([header['CD1_1'], header['CD1_2'],
                      header['CD2_1'], header['CD2_2']]).reshape(2,2)
    orientation = header['ORIENTAT']
    
    # Multiply by np.pi/180 if your orientation is in degrees 
    # instead of radians because the numpy functions want radians.
    CD1_1_rot = np.cos(-orientation*np.pi/180)
    CD1_2_rot = -np.sin(-orientation*np.pi/180)
    CD2_1_rot = np.sin(-orientation*np.pi/180)
    CD2_2_rot = np.cos(-orientation*np.pi/180)

    RotMat = np.array([CD1_1_rot, CD1_2_rot,
                      CD2_1_rot, CD2_2_rot]).reshape(2,2)

    RotMat_inv = np.array([CD1_1_rot, -CD1_2_rot,
                              -CD2_1_rot, CD2_2_rot]).reshape(2,2)

    # Apply rotation to the CDi_j header keywords.
    CDmat_rot = np.dot(CDmat,RotMat_inv)

    # Update header.
    header['CD1_1'], header['CD1_2'] = CDmat_rot[0]
    header['CD2_1'], header['CD2_2'] = CDmat_rot[1]
    header['ORIENTAT'] -= orientation

    # Rotate the image data. 
    # reshape=False is VERY important. If it is set to true, the 
    # dimensions of the array will change to accommodate the entire
    # rotated image. You'll have issues with the center value 
    # (CRVAL) and axis size (NAXIS) keywords. 
    rot_img = rotate(img, angle=orientation, reshape=False, 
        cval=np.nan)

    hdu[1].data = rot_img

    rot_wcs = WCS(header)
\end{minted}