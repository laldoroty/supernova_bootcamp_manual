\section{Software}
\label{sec:software}
\subsection{SALT}
\textcolor{red}{ADD SALT CITATIONS. ALL PAPERS. ALL OF THEM. AND WRITE WHAT THEY ARE}
\subsection{sncosmo}

\href{https://github.com/sncosmo/sncosmo}{sncosmo} has a whole bunch of contributors. It's a Python package, useful for middle steps of supernova cosmology (e.g., fitting models). 

\subsubsection{\texttt{helio\_to\_cmb()}}
\label{sec:heliotocmb}
The below code is handy for converting your heliocentric redshift to CMB redshift. Honestly, I can't find its location in the sncosmo github right now, so I've copied and pasted it below: \textcolor{red}{RUN THIS TO CHECK IT}
\begin{verbatim}
    import math
    from astropy.coordinates import SkyCoord
    
    # From sncosmo:
def radec_to_xyz(ra, dec):
    # SUPERNOVA BOOTCAMP MANUAL AUTHOR ADDITION: 
        Modified to add the try/except statement
    try:
        x = math.cos(np.deg2rad(dec)) 
            * math.cos(np.deg2rad(ra))
        y = math.cos(np.deg2rad(dec)) 
            * math.sin(np.deg2rad(ra))
        z = math.sin(np.deg2rad(dec))
    except:
        coord = SkyCoord('%s %s' % (ra, dec), unit=(u.hourangle,u.deg))
        x = math.cos(np.deg2rad(coord.dec.degree)) 
            * math.cos(np.deg2rad(coord.ra.degree))
        y = math.cos(np.deg2rad(coord.dec.degree)) 
            * math.sin(np.deg2rad(coord.ra.degree))
        z = math.sin(np.deg2rad(coord.dec.degree))

    return np.array([x, y, z], dtype=np.float64)

def cmb_dz(ra, dec):
    # See http://arxiv.org/pdf/astro-ph/9609034
    CMBcoordsRA = 167.98750000 # J2000 
    CMBcoordsDEC = -7.22000000

    # J2000 coords from NED\n",
    CMB_DZ = 371000. / 299792458.
    CMB_RA = 168.01190437
    CMB_DEC = -6.98296811
    CMB_XYZ = radec_to_xyz(CMB_RA, CMB_DEC)
    coords_xyz = radec_to_xyz(ra, dec)
    dz = CMB_DZ * np.dot(CMB_XYZ, coords_xyz)

    return dz

def helio_to_cmb(z, ra, dec):
    # Convert from heliocentric redshift to CMB-frame redshift.
    "    Parameters\n",
    "    ----------\n",
    "    z : float\n",
    "        Heliocentric redshift.\n",
    "    ra, dec: float\n",
    "        RA and Declination in degrees (J2000).\n",
    "    \"\"\"\n",

    dz = -cmb_dz(ra, dec)
    one_plus_z_pec = math.sqrt((1. + dz) / (1. - dz))
    one_plus_z_CMB = (1. + z) / one_plus_z_pec

    return one_plus_z_CMB - 1.
\end{verbatim}

\subsection{SNooPy}

\subsubsection{What does it do?}
\href{https://github.com/obscode/snpy}{SNooPy}, written in Python by Chris Burns for the Carnegie Supernova Project, has a lot of handy functions. \textcolor{red}{CITE}. 

Below, I'll list some of my favorite functions that are not-so-discussed in the \href{https://csp.obs.carnegiescience.edu/data/snpy/documentation}{documentation}. 

\subsubsection{\texttt{get\_dust\_RADEC()} and \texttt{get\_dust\_sigma\_RADEC()}}

\par
\texttt{get\_dust\_RADEC()} and \texttt{get\_dust\_sigma\_RADEC()} query \href{https://irsa.ipac.caltech.edu/frontpage/}{IRSA} using a given RA and dec to get the Milky Way $E(B-V)$ and error in $E(B-V)$, respectively. These are located in \texttt{snpy.utils.IRSA\_dust\_getval}. They accept arguments for RA and dec, with the default dust map set as \textcolor{red}{SF11 citation}. They return two things: the result and a flag. The flag indicates that the function worked. You can throw this out. 

\par

\noindent\textbf{Usage example:}
\textcolor{red}{CHECK THIS IS CORRECT, RUN IT:}
\begin{verbatim}
    from snpy.utils.IRSA_dust_getval import get_dust_RADEC, 
        get_dust_sigma_RADEC

    '''
    SN 1987A coordinates from 
    http://simbad.u-strasbg.fr/simbad/sim-id?Ident=SN+1987A.
    '''

    ra, dec = 279.703427, -31.937066
    mwreddening,_ = get_dust_RADEC(ra, dec, calibration='SF11')
    e_mwreddening,_ = get_dust_sigma_RADEC(ra, dec, calibration='SF11')
    mwreddening = mwreddening[0]
    e_mwreddening = e_mwreddening[0]

    print(f'The Milky Way reddening for SN 1987A is {mwreddening}
        +/- {e_mwreddening}.')

\end{verbatim}