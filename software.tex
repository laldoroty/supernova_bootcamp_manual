\section{Software and Models}
\label{sec:software}
\subsection{SALT}
\subsubsection{What is SALT?}


\subsubsection{Models}
\textit{SALT: a spectral adaptive light curve template for type Ia supernovae}, J. Guy et al. 2005 \cite{salt}. This is the first SALT model. The model is trained on 34 SNe with $z < 0.1$ in $UBVRI$ filters. \\

\textit{SALT2: using distant supernovae to improve the use of type Ia supernovae as distance indicators}, J. Guy et al. 2007 \cite{salt2}. SALT is retrained on data with $z \leq 1$. \\

\textit{Improved cosmological constraints from a joint analysis of the SDSS-II and SNLS supernova samples}, M. Betoule et al. 2014 \cite{salt2.4}. SALT2 is retrained on the data from \cite{salt2}, with the addition of SDSS-II data, for $z < 0.25$. SALT2.4 is born. \\

\textit{SALT3: An Improved Type Ia Supernova Model for Measuring Cosmic Distances}, W. D. Kenworthy et al. 2021 \cite{salt3}. SALT2.4 gets a facelift and becomes more open-sourcey. \\

\textit{SALT3-NIR: Taking the Open-source Type Ia Supernova Model to Longer Wavelengths for Next-generation Cosmological Measurements}, J. D. R. Pierel et al. 2022 \cite{salt3nir}. SALT3 gets upgraded to include NIR wavelengths. 

\subsubsection{How do I use SALT models?}

\subsection{sncosmo}
\href{https://github.com/sncosmo/sncosmo}{sncosmo} has a whole bunch of contributors. It's a Python package, useful for middle steps of supernova cosmology (e.g., fitting models). 

\subsubsection{Fitting a light curve}

\subsubsection{\texttt{helio\_to\_cmb()}}
\label{sec:heliotocmb}
The below code is handy for converting your heliocentric redshift to CMB redshift. Honestly, I can't find its location in the sncosmo github right now, so I've copied and pasted it below: 

\begin{minted}[
    bgcolor=lightgray,
    frame=leftline,
    framesep=-3mm]
    {python}
    
    import math
    import numpy as np
    from astropy.coordinates import SkyCoord
    
    # From sncosmo:
    def radec_to_xyz(ra, dec):
        # SUPERNOVA BOOTCAMP MANUAL AUTHOR ADDITION: 
        # Modified to add the try/except statement
        try:
            x = math.cos(np.deg2rad(dec)) 
                * math.cos(np.deg2rad(ra))
            y = math.cos(np.deg2rad(dec)) 
                * math.sin(np.deg2rad(ra))
            z = math.sin(np.deg2rad(dec))
        except:
            coord = SkyCoord('%s %s' % (ra, dec), 
                unit=(u.hourangle,u.deg))
            x = math.cos(np.deg2rad(coord.dec.degree)) 
                * math.cos(np.deg2rad(coord.ra.degree))
            y = math.cos(np.deg2rad(coord.dec.degree)) 
                * math.sin(np.deg2rad(coord.ra.degree))
            z = math.sin(np.deg2rad(coord.dec.degree))
    
        return np.array([x, y, z], dtype=np.float64)
    
    def cmb_dz(ra, dec):
        # See http://arxiv.org/pdf/astro-ph/9609034
        CMBcoordsRA = 167.98750000 # J2000 
        CMBcoordsDEC = -7.22000000
    
        # J2000 coords from NED\n",
        CMB_DZ = 371000. / 299792458.
        CMB_RA = 168.01190437
        CMB_DEC = -6.98296811
        CMB_XYZ = radec_to_xyz(CMB_RA, CMB_DEC)
        coords_xyz = radec_to_xyz(ra, dec)
        dz = CMB_DZ * np.dot(CMB_XYZ, coords_xyz)
    
        return dz

    def helio_to_cmb(z, ra, dec):
        # Convert from heliocentric redshift to CMB-frame redshift.
        "    Parameters\n",
        "    ----------\n",
        "    z : float\n",
        "        Heliocentric redshift.\n",
        "    ra, dec: float\n",
        "        RA and Declination in degrees (J2000).\n",
        "    \"\"\"\n",
    
        dz = -cmb_dz(ra, dec)
        one_plus_z_pec = math.sqrt((1. + dz) / (1. - dz))
        one_plus_z_CMB = (1. + z) / one_plus_z_pec
    
        return one_plus_z_CMB - 1.
\end{minted}

\subsection{SNooPy}

\subsubsection{What does it do?}
\href{https://github.com/obscode/snpy}{SNooPy} \cite{snoopy1,snoopy2}, written in Python by Chris Burns for the Carnegie Supernova Project, has a lot of handy functions. Below, I'll list some of my favorite functions that are not-so-discussed in the \href{https://csp.obs.carnegiescience.edu/data/snpy/documentation}{documentation}. 

\subsubsection{\texttt{get\_dust\_RADEC()} and \texttt{get\_dust\_sigma\_RADEC()}}

\par
\texttt{get\_dust\_RADEC()} and \texttt{get\_dust\_sigma\_RADEC()} query \href{https://irsa.ipac.caltech.edu/frontpage/}{IRSA} using a given RA and dec to get the Milky Way $E(B-V)$ and error in $E(B-V)$, respectively. These are located in \texttt{snpy.utils.IRSA\_dust\_getval}. They accept arguments for RA and dec, with the default dust map from \cite{sf11dust}. They return two things: the result and a flag. The flag indicates that the function worked. You can throw this out. \\

\noindent\textbf{Usage example:}
\begin{minted}[
    bgcolor=lightgray,
    frame=leftline,
    framesep=-3mm]
    {python}
    from snpy.utils.IRSA_dust_getval import get_dust_RADEC, 
        get_dust_sigma_RADEC

    '''
    SN 1987A coordinates from 
    http://simbad.u-strasbg.fr/simbad/sim-id?Ident=SN+1987A.
    Note: At the time of writing this, the get_dust_RADEC()
    function works, but get_dust_sigma_RADEC() is failing.
    Otherwise, this code works as-is.
    '''

    ra, dec = 279.703427, -31.937066
    mwreddening,_ = get_dust_RADEC(ra, dec, calibration='SF11')
    e_mwreddening,_ = get_dust_sigma_RADEC(ra, dec, 
        calibration='SF11')
    mwreddening = mwreddening[0]
    e_mwreddening = e_mwreddening[0]

    print(f'The Milky Way reddening for SN 1987A is {mwreddening}
        +/- {e_mwreddening}.')

\end{minted}