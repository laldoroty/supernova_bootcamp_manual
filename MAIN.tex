%% Overleaf			
%% Software Manual and Technical Document Template	
%% 									
%% This provides an example of a software manual created in Overleaf.

\documentclass{ol-softwaremanual}

% Packages used in this example
\usepackage{graphicx}  % for including images
\usepackage{xcolor}
\usepackage{microtype} % for typographical enhancements
\usepackage{minted}    % for code listings
\usepackage{amsmath}   % for equations and mathematics
\setminted{style=friendly,fontsize=\small}
\renewcommand{\listoflistingscaption}{List of Code Listings}
\usepackage{hyperref}  % for hyperlinks
\usepackage[a4paper,top=3.5cm,bottom=3.5cm,left=3.5cm,right=3.5cm]{geometry} % for setting page size and margins

% Custom macros used in this example document
\newcommand{\doclink}[2]{\href{#1}{#2}\footnote{\url{#1}}}
\newcommand{\cs}[1]{\texttt{\textbackslash #1}}

% Frontmatter data; appears on title page
\title{Supernova Bootcamp}
\version{0.0.0}
\author{Lauren Aldoroty}
%\softwarelogo{\includegraphics[width=8cm]{logo}}

\begin{document}

\maketitle

\tableofcontents
% \listoflistings
\newpage

\section{Introduction}

Compiling all the things I've learned so it's easier for future folks. This manual will attempt to cover more technical, data-analysis driven perspectives on working with SNe, particularly SNe Ia. In the Software section (Section~\ref{sec:software}), at the end of each subsection, I'll list some lesser-known functions and tricks that I've found useful. These will be indicated by section titles \texttt{in font like this}, e.g., \texttt{helio\_to\_cmb()} (Section~\ref{sec:heliotocmb}).

\section{Reddening and Extinction}

Reddening and extinction are \textit{nearly} the same thing, but not quite. They're both caused by light scattering off dust, but \textit{extinction} describes the dimming that results, and \textit{reddening} describes the preferential scattering of blue light (i.e., more shorter wavelength light gets scattered away from us than longer wavelengths, so we see objects as redder than they actually are.) This is important because it's a pretty large source of systematic uncertainty in supernova studies! Dust doesn't emit light, it just takes it away, so we can't ``see'' it directly. \textit{HOW} are we supposed to correct for an effect we know so little about? 

\subsection{Supernova Intrinsic Color}

\subsection{Correcting for Dust}
\section{Hubble's Law and SN Ia Cosmology}

\subsection{What the heck is a rest frame?}

Let's say you have some data, but it's from a supernova with $z = 0.1$. Because of the relativistic Doppler effect, the wavelengths you \textit{observe} are different from the ``actual'' wavelengths, i.e., rest frame wavelengths. This will also affect observed flux. Ignoring ejecta velocity (see \textcolor{red}{CHAPTER FOR THIS}), this is why a spectral line won't appear in its ``expected'', or rest frame, position. Usually, you want to put your spectra in the rest frame before doing anything with them because it's hard to compare SNe at different redshifts. See Section~\ref{sec:spec_restframe} for details on correcting SN spectra to the rest frame. 

\subsection{$z_{helio}$ and $z_{CMB}$}

You'll encounter two kinds of redshift---heliocentric redshift ($z_{helio}$) and CMB (Cosmic Microwave Background) redshift ($z_{CMB}$). \textit{These are different, and are used for different purposes!} $z_{CMB}$ is the redshift caused by the Universe's expansion \textit{only}, i.e., where the reference frame is the CMB frame. $z_{helio}$ is the redshift with \textit{only} the Earth's rotation and orbit removed. There are still effects from other motion, like Galactic rotation and the Galaxy moving around with respect to other objects, as well as $z_{CMB}$. ``Heliocentric'' $\equiv$ ``Sun at center'', so ``Sun at center'' is the rest frame. It's how things are moving relative to the Milky Way. Note that $z_{helio}$, while often reported as an object's redshift, is \textit{not} the observed redshfit. The observed redshift does not have the Earth's rotation and orbit removed. In other words, $z_{helio}$ contains information about a bunch of things moving relative to each other, including motion relative to the CMB, while $z_{CMB}$ contains information about \textit{only} motion relative to the CMB.

For cosmology, you'll use $z_{CMB}$ (e.e., when making a Hubble diagram or determining other cosmological parameters). If you're dealing with data that you need to correct to the rest frame, you probably want to use $z_{helio}$.
\section{Photometry}
Caveat about this section: I have only ever used the Vega magnitude system. There will be no discussion of AB magnitudes. 

\subsection{Converting from flux to magnitude}
Here, we just use the definition of magnitudes:

\begin{equation}
    m = -2.5\log_{10} \Big( \frac{F_{obs}}{F_{ref}} \Big),
\end{equation}

where $F_{obs}$ is your observed flux and $F_{ref}$ is your reference flux in the \textit{same} photometric band. You need to make sure your units are consistent between $F_{obs}$ and $F_{ref}$. \textit{You need to understand your magnitude system before doing this.} I repeat, \textit{YOU NEED TO UNDERSTAND YOUR MAGNITUDE SYSTEM BEFORE DOING ANYTHING!} Know your units. Know your initial units, know your end units. Know the units of your reference, know the units of your data. 

Using equation \ref{eqn:errorprop}, we can also convert flux error to magnitude error:
\begin{equation}
    \label{eqn:flux-to-mag-err}
    \sigma_{m} = \sqrt{\Big( \frac{2.5}{\ln10} \Big)^{2} \Big( \frac{\sigma_{F_{obs}}}{F_{obs}}\Big)^{2}}
\end{equation}

\subsection{Converting from magnitude to flux}
\label{sec:magtoflux}
Rearrange the definition of magnitudes:
\begin{equation}
    F_{obs} = F_{ref}10^{\frac{-m}{2.5}}
\end{equation}

Then, if you want to convert the magnitude error to flux error, use equation \ref{eqn:errorprop} on this equation, and get: 

\begin{equation}
    \sigma_{F_{obs}} = \sqrt{\Big( \sigma_{m} \frac{F_{ref}\ln(10)}{2.5} 10^{\frac{-m}{2.5}} \Big)^{2}}
\end{equation}

I'd like to issue a word of caution on converting from magnitude units back to flux units. If you convert a magnitude in a particular magnitude system, e.g. Vega, back to flux units, then the flux you've computed is the flux of the object \textit{in the Vega system, not the physical flux of the source}. This first came up for me when using \texttt{sncosmo}, so if you're having some zeropoint troubles with that code and you converted from magnitudes to flux, this is likely the issue. Check out \ref{sec:sncosmo} for the procedure to fix this. 
\section{Spectra}
\subsection{Correcting to the rest frame}
\label{sec:spec_restframe}
Correcting the observed wavelength for the relativistic Doppler effect is done by: 
\begin{equation}
    \lambda_{rest \,\, frame} = \frac{\lambda_{observed}}{1+z_{helio}}.
\end{equation}

Correcting the observed flux is more complicated and I've redacted this section for now. If you need to correct flux to the rest frame, see the next section and normalize to a chosen $z$. 

\subsection{Normalizing to a particular $z$}
Sometimes, you want your spectrum flux as if it were at some redshift $z_{ref}$ that you've chosen. Do not use the rest frame spectrum for this calculation. First thing's first, you need the luminosity distance to your object.

\begin{equation}
    D_{L} = (1+z_{helio})D_{M},
\end{equation}

where $z_{helio}$ is the heliocentric ($\sim$observed) redshift for your object, and $D_{M}$ is the comoving transverse distance. In Python, you can use \texttt{astropy.cosmology}'s function \texttt{comoving\_transverse\_distance}, which takes $z_{CMB}$ as its input (so you'll also need your object's RA and dec), to calculate $D_{M}$. If we assume an FLRW universe, then our cosmological scale factor is $a(t) = 1/(1+z)$. Then,
\begin{equation}
    \frac{a(t_{ref})}{a(t_{actual})} = \Big( \frac{1 + z_{helio}}{1+z_{ref}} \Big) \Big( \frac{D_{L, actual}}{D_{L, ref}} \Big)^{2}
\end{equation}

Finally, convert your flux by using this value:
\begin{equation}
    F_{ref} = F_{actual}\Big( \frac{1 + z_{helio}}{1+z_{ref}} \Big) \Big( \frac{D_{L, actual}}{D_{L, ref}} \Big)^{2} = F_{actual} \Big(\frac{a(t_{ref})}{a(t_{actual})} \Big)
\end{equation}

Using Equation~\ref{eqn:errorprop}, the error is:
\begin{equation}
    \sigma_{F_{ref}} = \sigma_{F_{actual}}\frac{\partial F_{ref}}{\partial F_{actual}} = \sigma_{F_{actual}}\Big( \frac{1 + z_{helio}}{1+z_{ref}} \Big) \Big( \frac{D_{L, actual}}{D_{L, ref}} \Big)^{2}
\end{equation}

\subsection{Absorption lines}
\subsubsection{Pseudo-equivalent width}
Pseudo-equivalent width (pEW) is a measure of the strength of an absorption line. It combines the depth and the width of the line into one measurement by integrating over the feature. There's a really good run-down of pEW in \cite{Galbany2015}. So, we're going to borrow from them. Figure~\ref{fig:pew} shows an absorption feature. You start with one like the one on the left (red). We want to measure the strength of the absorption feature, but the problem with SNe is that there are so many absorption and emission lines and there's doppler shift everywhere, so we don't really know where the continuum is. \textit{But}, we need a way to fairly compare absorption features between different SNe, which means removing the continuum. So, we make a ``pseudo-continuum'' to approximate a small region of the actual continuum. Lines are nice, so all you do to make the pseudo-continuum is draw a line across the top of the absorption feature. We remove the ``continuum'' by normalizing the absorption feature to the pseudo-continuum. You only do this for the one feature---for each absorption feature, you need a new pseudo-continuum. After you've normalized to the pseudo-continuum, you end up with the \textit{right} panel of Figure~\ref{fig:pew} (green). Now, you can measure the area and fairly compare line strengths between SNe. 

\begin{figure}[h!]
    \centering
    \includegraphics[width=0.9\textwidth]{figs/Screenshot from 2022-07-12 16-05-37.png}
    \caption{\textit{Left}: Un-normalized absorption feature. \textit{Right}: The same absorption feature normalized to the pseudo-continuum marked on the \textit{left}. $d$ is the feature depth and $\lambda_{e}$ is the center wavelength. The shaded area is the pEW (pW in the figure). Figure from \cite{Galbany2015}.}
    \label{fig:pew}
\end{figure}

That's great and all, but how do we actually do all this in practice? There are a lot of ways. Let's break it down into steps and talk about 'em. This is \textit{not} an exhaustive list, nor is it the only way to do this---these are merely suggestions. Let your imagination run free, spectral padawan.

\begin{enumerate}
    \item \textbf{Finding the pseudo-continuum:} The challenging part of this step is avoiding noise. We don't want an unrealistic pseudo-continuum because we picked a point where there's a statistical spike or dip in the measurement. You can smooth the data with a moving average, or use the Savitzky-Golay filter from \texttt{scipy.signal.savgol\_filter()}. Discussing the differences between these two is outside the scope of this manual, but you should always understand the techniques you're using! 
    \item \textbf{Normalizing to the pseudo-continuum:} Take your smoothed flux and divide by the line you drew. Here's some Python-inspired pseudo-code to help you out:
    \begin{minted}[
    bgcolor=lightgray,
    frame=leftline,
    framesep=-3mm]
    {python}
    continuum = scipy.interpolate.interp1d(flux[start_point],
        flux[end_point])
    normalized_flux = flux[start_point:end_point]
        /continuum(wavelength[start_point:end_point])
    \end{minted}
    
    \item \textbf{Integrating over the absorption feature:} You can do this via summation directly over the normalized data or by fitting a Gaussian and integrating that. 
\end{enumerate}

Now... we weren't going to have this entire discussion without talking about error on pEW, of course. The error for this kind of ``guesstimation'' thing is hard to quantify precisely, so I recommend a bootstrapping method like the one in \cite{Galbany2015}. Instead of choosing fixed endpoints for your pseudo-continuum, draw randomly from some small region around your chosen endpoints $N$ times to get $N$ sets of different, but all reasonable, endpoints. Now, calculate the pEW as discussed above for each set of endpoints. Then, your final pEW measurement is the mean of all $N$ measurements, with the standard deviation of all $N$ measurements as your error on pEW. 

\subsubsection{Ejecta velocity}
\label{sec:ejectavelocity}

\subsection{Synthetic photometry}

If you're here, you're wondering how to make photometry from some spectra. In short, this is obtained by integrating under the spectrum in the region covered by a given filter. 

The area under the spectrum, $F$, in a given filter $X$, is calculated by
\begin{equation}
    F_{X} = \frac{1}{hc}\int_{\lambda_{1}}^{\lambda_{2}} \lambda f(\lambda) R_{X}(\lambda) d\lambda,
\label{eqn:synthint}
\end{equation}

where $f(\lambda)$ is your spectrum, and $R_{X}(\lambda)$ is your filter. The $\lambda$ and $1/hc$ terms are in there if we're using photon-counting detectors, and using an input spectrum that's in energy units. If you're using an energy-counting detector, you would drop this, i.e., your integrand is $f(\lambda) R_{X}(\lambda) d\lambda$.

However, we can't often use the continuous version of this function with our data. So, we discretize it:

\begin{equation}
    F_{X} = \frac{1}{hc}\sum_{i=\lambda_{1}}^{\lambda_{2}} \lambda_{i} f(\lambda_{i}) R_{X}(\lambda_{i})(\lambda_{i} - \lambda_{i-1}),
\label{eqn:synth_discrete}
\end{equation}

where $f(\lambda_{i})$ is the energy flux at a particular wavelength $\lambda_{i}$, and $R_{X}(\lambda_{i})$ is the response function for the filter at wavelength $\lambda_{i}$. We're using a discretized version of an integral, so $(\lambda_{i} - \lambda_{i-1})$ is the same as $d\lambda$. From now on, this chapter will be written as if we are using Equation \ref{eqn:synth_discrete}.

Then, for a photon-counting detector with an input spectrum in units of energy, the variance of $F$ (using Equation \ref{eqn:errorprop}) is:

\begin{equation}
\label{eqn:varphot}
    \sigma_{F}^{2} = \frac{1}{(hc)^{2}}\sum_{i=\lambda_{1}}^{\lambda_{2}} \lambda_{i}^{2}\sigma_{f(\lambda_{i})}^{2} R_{X}(\lambda_{i})^{2}(\lambda_{i} - \lambda_{i-1})^{2} 
\end{equation}

For an energy-counting detector with an input spectrum in units of energy, the variance of $F$ is:
\begin{equation}
\label{eqn:varenergy}
    \sigma_{F}^{2} = \sum_{i=\lambda_{1}}^{\lambda_{2}} \sigma_{f(\lambda_{i})}^{2} R_{X}(\lambda_{i})^{2}(\lambda_{i} - \lambda_{i-1})^{2} 
\end{equation}

Individual magnitudes are calculated by
\begin{equation}
    m = -2.5\log_{10}\Big(\frac{F}{F_{ref}}\Big) = -2.5\log_{10}(F) + 2.5\log_{10}(F_{ref}) = -2.5\log_{10}(F) - m_{ref}.
\end{equation}
\textit{Wait---what are $F_{ref}$ and $m_{ref}$}? Good question. Because magnitudes are inherently relative quantities, we need to use some reference object to convert flux into magnitudes. You can get reference objects from \href{https://www.stsci.edu/hst/instrumentation/reference-data-for-calibration-and-tools/astronomical-catalogs/calspec}{CALSPEC}. Photometric systems are, unfortunately, out of the scope of this manual at this time. Let's say we're using the Vega system, with the star Vega (Alpha Lyrae) from now on. 

This means we need to use Equation \ref{eqn:synth_discrete} on our reference object, Vega to calculate $F_{ref}$. You'll use the \textit{same} response function as you use for your supernova spectrum. This means that $m_{ref}$ is the... reference magnitude? What does THAT mean? If magnitudes are inherently relative quantities, do we then compare our reference object to something \textit{else}? A valid concern, but no! This is called a \textit{zero point}. For the Vega system, astronomers have defined the magnitude of the star Vega such that the $m_{Vega}$ = 0 in all bands. So, if you're using the Vega system, $m_{ref}$ is probably just 0 (unless some literature tells you otherwise, which is possible, so make sure you're familiar with the photometric system and filters you're using).

So anyway, the variance of $m$ (for an energy-counting detector) is:
\begin{equation}
    \sigma_{m}^{2} = \frac{\partial m}{\partial f}^{2} \sigma_{f}^{2} = \frac{\partial m}{\partial F}^{2} \Big( \sum_{i=\lambda_{1}}^{\lambda_{2}} \frac{\partial F}{\partial f(\lambda_{i})} \sigma_{f(\lambda_{i})}\Big)^{2} = \Big( \frac{-2.5}{F\ln10} \Big)^{2} \sum_{i=\lambda_{1}}^{\lambda_{2}}\sigma_{f(\lambda_{i})}^{2} R_{X}(\lambda_{i})^{2}(\lambda_{i}-\lambda_{i-1})^{2}
    \label{eqn:m_err}
\end{equation}

Note that the summation term on the right-hand side of the equation is the same as $\sigma_{F}^{2}$. You need to be careful about flux variance here, though: Equations~\ref{eqn:synth_discrete},   \ref{eqn:varphot}, and \ref{eqn:varenergy} are true, but when you are calculating magnitudes, keep in mind that you're calculating this based off a relative flux. So, your flux variance should also be relative. In other words, it should be the variance of $F/F_{ref}$. Using Equation~\ref{eqn:errorprop} and treating $F_{ref}$ as a constant, we get

\begin{equation}
    \sigma^{2}_{F/F_{ref}} = \frac{1}{F_{ref}^{2}}\sum_{i=\lambda_{1}}^{\lambda_{2}} \lambda_{i}\sigma_{f(\lambda_{i})}^{2} R_{X}(\lambda_{i})^{2}(\lambda_{i} - \lambda_{i-1})^{2} 
\end{equation}

(or the equivalent energy-counting version of this). Note that we do not account for error on $F_{ref}$, here. That's why we just use $\sigma_{f(\lambda_{i})}$, which is the error of your data spectrum. \\

If you're looking for \textit{colors}, I've got your back there, too. For arbitrary color $X-Y$,
\begin{equation}
    m_{X} - m_{Y} = -2.5\log \Big( \frac{F_{X}}{F_{ref,X}} \Big) + 2.5\log \Big( \frac{F_{Y}}{F_{ref,Y}} \Big).
\end{equation}
We can propagate the error here, too. If you want, you can assume $X$ and $Y$ are independent, and do $\sigma_{X-Y} = \sqrt{\sigma_{X}^{2} + \sigma_{Y}^{2}}$. However, we can be more rigorous and not assume independence. We're going to use $\sigma^{2} = \mathbf{JCJ^{T}}$---the explanation of this formula is beyond the scope of this manual. $\mathbf{J}$ is the Jacobian of your spectrum (remember, your spectrum is a function!), and $\mathbf{C}$ is its covariance matrix. The following method will work \textit{if you have an error for the flux in your data spectrum}.

For ease of calculations, we will treat $m_{X}-m_{Y}$ as a function of $f(\lambda_{i})$ (our supernova spectrum, in discrete wavelength chunks). Then, for arbitrary color $X-Y$, the Jacobian is:
\begin{equation}
\label{eqn:jac2}
\mathbf{J}_{X-Y} = 
\begin{bmatrix}
    
        \frac{\partial m_{X-Y}}{\partial f(\lambda_{0})} & \frac{\partial m_{X-Y}}{\partial f (\lambda_{1})} & \dots & \frac{\partial m_{X-Y}}{\partial f(\lambda_{N})} \\
    
\end{bmatrix}
\end{equation}

\parindent = 0 mm

where $N$ is the last measured wavelength in the spectrum. The $i$th entry is: 
\begin{equation}
    \frac{\partial m_{X-Y}}{\partial f(\lambda_{i})} = -\frac{1.09}{F_{X}} R_{X}(\lambda_{i})(\lambda_{i}-\lambda_{i-1}) + \frac{1.09}{F_{Y}}R_{Y}(\lambda_{i})(\lambda_{i}-\lambda_{i-1})
\end{equation}

Then, the covariance matrix is diagonal, and each entry is the spectrum error provided in the data:
\begin{equation}\mathbf{C_{X-Y}} = 
    \begin{bmatrix}
        \sigma_{f(\lambda_{0})}^{2} &  & & \\
         & \sigma_{f(\lambda_{1})}^{2}&  & \\
         &  & \ddots &  \\
        & & & & \sigma_{f(\lambda_{N})}^{2}\\
    \end{bmatrix}.
    \label{eqn:cov}
\end{equation}
Now, we use $\sigma^{2} = \mathbf{JCJ^{T}}$, and boom, we have color error without assuming independence for the filters. 

\subsection{Things to watch out for}
\subsubsection{Response function units}
Your response function will likely be normalized, but may be given in either normalized flux units or normalized photon counts. You need to know which units you have. If you need the other, don't fret---you can convert it to the other unit system. Let's say you have a response function in normalized photon units, but your spectrum is in energy flux units (ergs/cm$^{2}$/s/$\mathrm{\AA}$). It's best to convert the spectrum to photons. For each $i$th wavelength, you do:

\begin{equation}
    f_{\gamma}(\lambda_{i}) = \frac{\lambda_{i}}{hc}f_{E}(\lambda_{i}),
\end{equation}
where $h \approx 6.626 \times 10^{-27}$ ergs $\cdot$ s and $c \approx 3\times 10^{18}$ is the speed of light in $\mathrm{\AA}$/s. 

Why convert the spectrum instead of the response function? Well, CCDs count photons, so it's ideal to do as much as possible in these units. \\

Don't forget to convert your errors, as well. Using Equation~\ref{eqn:errorprop},

\begin{equation}
    \sigma_{f_{\gamma}(\lambda_{i})} = \frac{\lambda_{i}}{hc}\sigma_{f_{E}(\lambda_{i})}.
\end{equation}

\subsubsection{Help! My results are unreasonable!}

Did you check all of the following:
\begin{itemize}
    \item Did you convert both your standard spectrum \textit{and} your data spectrum from ergs to photons (or vice versa)?
    \item When you converted between ergs and photons, did you use the correct units for the constants? Remember, for ergs/cm$^{2}$/s/$\mathrm{\AA}$, use $h \approx 6.626 \times 10^{-27}$ ergs $\cdot$ s and $c \approx 3\times 10^{18}$ $\mathrm{\AA}$/s. 
    \item When you converted between ergs and photons, did you multiply by the conversion factor when you should have divided (or vice versa)? 
    \item Does the wavelength range of your filter fall completely inside the wavelength range of your spectrum? 
    \item Did you convert units when maybe you shouldn't have because you already did it? 
\end{itemize}

\section{Classifying spectra}
You can use \href{https://people.lam.fr/blondin.stephane/software/snid/}{SNID} or the in-browser option \href{https://gelato.tng.iac.es/gelato/}{GELATO}. Tutorials forthcoming. 
\section{Software and Models}
\label{sec:software}
\subsection{SALT}
\subsubsection{What is SALT?}


\subsubsection{Models}
\textit{SALT: a spectral adaptive light curve template for type Ia supernovae}, J. Guy et al. 2005 \cite{salt}. This is the first SALT model. The model is trained on 34 SNe with $z < 0.1$ in $UBVRI$ filters. \\

\textit{SALT2: using distant supernovae to improve the use of type Ia supernovae as distance indicators}, J. Guy et al. 2007 \cite{salt2}. SALT is retrained on data with $z \leq 1$. \\

\textit{Improved cosmological constraints from a joint analysis of the SDSS-II and SNLS supernova samples}, M. Betoule et al. 2014 \cite{salt2.4}. SALT2 is retrained on the data from \cite{salt2}, with the addition of SDSS-II data, for $z < 0.25$. SALT2.4 is born. \\

\textit{SALT3: An Improved Type Ia Supernova Model for Measuring Cosmic Distances}, W. D. Kenworthy et al. 2021 \cite{salt3}. SALT2.4 gets a facelift and becomes more open-sourcey. \\

\textit{SALT3-NIR: Taking the Open-source Type Ia Supernova Model to Longer Wavelengths for Next-generation Cosmological Measurements}, J. D. R. Pierel et al. 2022 \cite{salt3nir}. SALT3 gets upgraded to include NIR wavelengths. 

\subsubsection{How do I use SALT models?}

\subsection{sncosmo}
\href{https://github.com/sncosmo/sncosmo}{sncosmo} has a whole bunch of contributors. It's a Python package, useful for middle steps of supernova cosmology (e.g., fitting models). 

\subsubsection{Fitting a light curve}

\subsubsection{\texttt{helio\_to\_cmb()}}
\label{sec:heliotocmb}
The below code is handy for converting your heliocentric redshift to CMB redshift. Honestly, I can't find its location in the sncosmo github right now, so I've copied and pasted it below: 

\begin{minted}[
    bgcolor=lightgray,
    frame=leftline,
    framesep=-3mm]
    {python}
    
    import math
    import numpy as np
    from astropy.coordinates import SkyCoord
    
    # From sncosmo:
    def radec_to_xyz(ra, dec):
        # SUPERNOVA BOOTCAMP MANUAL AUTHOR ADDITION: 
        # Modified to add the try/except statement
        try:
            x = math.cos(np.deg2rad(dec)) 
                * math.cos(np.deg2rad(ra))
            y = math.cos(np.deg2rad(dec)) 
                * math.sin(np.deg2rad(ra))
            z = math.sin(np.deg2rad(dec))
        except:
            coord = SkyCoord('%s %s' % (ra, dec), 
                unit=(u.hourangle,u.deg))
            x = math.cos(np.deg2rad(coord.dec.degree)) 
                * math.cos(np.deg2rad(coord.ra.degree))
            y = math.cos(np.deg2rad(coord.dec.degree)) 
                * math.sin(np.deg2rad(coord.ra.degree))
            z = math.sin(np.deg2rad(coord.dec.degree))
    
        return np.array([x, y, z], dtype=np.float64)
    
    def cmb_dz(ra, dec):
        # See http://arxiv.org/pdf/astro-ph/9609034
        CMBcoordsRA = 167.98750000 # J2000 
        CMBcoordsDEC = -7.22000000
    
        # J2000 coords from NED\n",
        CMB_DZ = 371000. / 299792458.
        CMB_RA = 168.01190437
        CMB_DEC = -6.98296811
        CMB_XYZ = radec_to_xyz(CMB_RA, CMB_DEC)
        coords_xyz = radec_to_xyz(ra, dec)
        dz = CMB_DZ * np.dot(CMB_XYZ, coords_xyz)
    
        return dz

    def helio_to_cmb(z, ra, dec):
        # Convert from heliocentric redshift to CMB-frame redshift.
        "    Parameters\n",
        "    ----------\n",
        "    z : float\n",
        "        Heliocentric redshift.\n",
        "    ra, dec: float\n",
        "        RA and Declination in degrees (J2000).\n",
        "    \"\"\"\n",
    
        dz = -cmb_dz(ra, dec)
        one_plus_z_pec = math.sqrt((1. + dz) / (1. - dz))
        one_plus_z_CMB = (1. + z) / one_plus_z_pec
    
        return one_plus_z_CMB - 1.
\end{minted}

\subsection{SNooPy}

\subsubsection{What does it do?}
\href{https://github.com/obscode/snpy}{SNooPy} \cite{snoopy1,snoopy2}, written in Python by Chris Burns for the Carnegie Supernova Project, has a lot of handy functions. Below, I'll list some of my favorite functions that are not-so-discussed in the \href{https://csp.obs.carnegiescience.edu/data/snpy/documentation}{documentation}. 

\subsubsection{\texttt{get\_dust\_RADEC()} and \texttt{get\_dust\_sigma\_RADEC()}}

\par
\texttt{get\_dust\_RADEC()} and \texttt{get\_dust\_sigma\_RADEC()} query \href{https://irsa.ipac.caltech.edu/frontpage/}{IRSA} using a given RA and dec to get the Milky Way $E(B-V)$ and error in $E(B-V)$, respectively. These are located in \texttt{snpy.utils.IRSA\_dust\_getval}. They accept arguments for RA and dec, with the default dust map from \cite{sf11dust}. They return two things: the result and a flag. The flag indicates that the function worked. You can throw this out. \\

\noindent\textbf{Usage example:}
\begin{minted}[
    bgcolor=lightgray,
    frame=leftline,
    framesep=-3mm]
    {python}
    from snpy.utils.IRSA_dust_getval import get_dust_RADEC, 
        get_dust_sigma_RADEC

    '''
    SN 1987A coordinates from 
    http://simbad.u-strasbg.fr/simbad/sim-id?Ident=SN+1987A.
    Note: At the time of writing this, the get_dust_RADEC()
    function works, but get_dust_sigma_RADEC() is failing.
    Otherwise, this code works as-is.
    '''

    ra, dec = 279.703427, -31.937066
    mwreddening,_ = get_dust_RADEC(ra, dec, calibration='SF11')
    e_mwreddening,_ = get_dust_sigma_RADEC(ra, dec, 
        calibration='SF11')
    mwreddening = mwreddening[0]
    e_mwreddening = e_mwreddening[0]

    print(f'The Milky Way reddening for SN 1987A is {mwreddening}
        +/- {e_mwreddening}.')

\end{minted}
\newcommand{\cov}{\mathrm{cov}}

\section{Statistics Stuff}
I guess you could argue this is outside the scope of a supernova manual, but these are things I use all the time. Plus, it's my darn manual, so I get to put what I want in here.

\subsection{Weighted expectation value, variance, and covariance}
\textbf{Expectation value}, unweighted by data error:
\begin{equation}
\label{eqn:expval}
    E[X] = \sum_{i} x_{i}p_{i},
\end{equation}
where $x_{i}$ is your data and $p_{i}$ is the probability of that value. Now, if you want to use your data's error instead of the probability (which is hard/weird to quantify):
\begin{equation}
\label{eqn:weightedexpval}
    E_{w}[X] = \frac{\sum_{i}x_{i}/(\sigma_{x_{i}}^{2})}{\sum_{i} (1/\sigma_{x_{i}}^{2})},
\end{equation}
where $\sigma_{x_{i}}$ is the error for data point $x_{i}$.

Great, then what's the \textbf{variance of a weighted expectation value} (mean)? 

Unweighted, variance is defined as:
\begin{equation}
    \mathrm{Var}[X] = E[(X-\mu)^{2}] = \sum_{i} p_{i}(x_{i}-\mu)^{2},
\end{equation}
where $\mu$ is the mean of the data and $p$ is the probability of that $x_{i}$ occurring. So, we calculate the weighed variance with this formula, but using the weighted mean (a.k.a. expectation value) from Equation \ref{eqn:weightedexpval}, and change $p_{i}$ to $w_{i} = 1/\sigma_{i}^{2}$:

\begin{equation}
\label{eqn:weightedvariance}
    \mathrm{Var}_{w}[X] = E_{w}[(X - \mu_{w})^{2}] = \frac{1}{\sum_{i} \sigma_{x_{i}}^{2}} \sum_{i}\frac{1}{\sigma_{x_{i}}^{2}} (x_{i} - E_{w}[X])^{2})
\end{equation}

Now, on to \textbf{weighted covariance}. Covariance is defined as 

\begin{equation}
    \mathrm{cov}(X,Y) = E[(X-E[X])(Y-E[Y])]
\end{equation}

So, our weighted covariance is going to be this formula, but using weighted variance and expectation values:

\begin{equation}
    \mathrm{cov}_{w}(X,Y) = E[(X-E_{w}[X])(Y-E_{w}[Y])]
\end{equation}

The tricky part here is how we handle the errors. Using the error propagation formula (Equation \ref{eqn:errorprop}) on $(X-E[X])(Y-E[Y])$ and then taking the reciprocal of this, we can weight by 

\begin{equation}
    w_{i} = \frac{1}{(y_{i}-E_{w}[Y])^{2}\sigma_{x_{i}}^{2} + (x_{i}-E_{w}[X])^{2}\sigma_{y_{i}}^{2}}
\end{equation}
Thus, weighted covariance is
\begin{equation}
    \mathrm{cov}_{w} = \frac{1}{\sum_{i}w_{i}}\sum_{i}w_{i}(x_{i}-E_{w}[X])(y_{i}-E_{w}[Y])
\end{equation}

\subsection{Weighted Pearson correlation coefficient}
The Pearson correlation coefficient measures how linearly correlated two datasets are. It ranges from $[-1,1]$, where -1 is perfectly negatively linearly correlated, 1 is perfectly positively linearly correlated, and 0 is no correlation. Your correlation coefficient, weighted or unweighted, is:
\begin{equation}
\label{eqn:corrcoeff}
    \rho_{X,Y} = \frac{\cov(X,Y)}{\sigma_{X}\sigma_{Y}},
\end{equation}
where $\cov(X,Y)$ is the covariance between datasets $X$ and $Y$, and $\sigma$ is the error for a dataset. In order to make this weighted, you calculate the weighted covariance for the numerator, and use weighted variances for the denominator. In terms of things we've discussed earlier in this chapter:

\begin{equation}
    \rho_{X,Y,w} = \frac{\mathrm{cov}_{w}(X,Y)}{\sqrt{\mathrm{Var}_{w}(X)\mathrm{Var}_{w}(Y)}}
\end{equation}

That's great and all, but what about the significance of this? What's the $p$-value? We can calculate this with a two-sided $t$-test, where the test statistic is
\begin{equation}
    t = \frac{\rho_{X,Y}\sqrt{N-2}}{\sqrt{1-\rho_{X,Y}^{2}}}.
\end{equation}
$\rho_{X,Y}$ is your Pearson correlation coefficient (weighted or unweighted), $N$ is the number of data points, and 2 represents the free parameters in the fit (i.e., $N-2$ is the degrees of freedom in the fit). To get the $p$-value, you have three options: 1. Do a painful integral 2. Use a table 3. Use \texttt{scipy.special.stdtr} in Python, which does the painful integral for you. I will not be discussing options 1 and 2, because they stink. Option 3 is best. Use \texttt{stdtr()} like this: \texttt{2*stdtr(dof,-|t|)}, where \texttt{dof} is your degrees of freedom, ($N-2$), and \texttt{t} is the test statistic. You slap the absolute value signs on and multiply the final $p$-value by 2 because you're integrating between two values in the $t$ distribution.

\subsection{Error Propagation}
\label{sec:errorprop}
In order to propagate error, you need to know the functions that describe your model, because you're going to have to take derivatives. The simplest way to propagate error for a function with $n$ variables is: $f(x_{0},x_{1}, ...,  x_{n})$ is:
\begin{equation}
\label{eqn:errorprop}
    \sigma_{f} = \sqrt{\sum_{i=0}^{n}\Big( \frac{\partial f}{\partial x_{i}} \sigma_{x_{i}}\Big)^{2}},
\end{equation}
where $\sigma_{x_{i}}$ is the error for variable $x_{i}$. Note that this formula \textit{assumes the variables are all independent}. 

\subsection{Least Squares and Minimizing $\chi^{2}$}
In general, when minimizing $\chi^{2}$ to fit a model, 

\begin{equation}
    \chi^{2} = \sum_{i=0}^{k} \Big( \frac{X_{i} - \mu_{i}}{\sigma_{i}} \Big)^{2}
\end{equation}
where $X_{i}$ is the data, $\mu_{i}$ is the model, and $\sigma_{i}$ is the error for the data. $k$ is the number of data points. \\

Now, use Python to minimize this function. You can use any minimization algorithm you want. Personally, I like least squares. You can use \href{https://github.com/cosmonaut/pycmpfit}{\texttt{pycmpfit}} or \href{https://github.com/segasai/astrolibpy/blob/master/mpfit/mpfit.py}{\texttt{mpfit}} (see Section \ref{sec:mpfit}). Both will work, but \href{https://github.com/cosmonaut/pycmpfit}{\texttt{pycmpfit}} is installable via pip. If you want to use \href{https://github.com/segasai/astrolibpy/blob/master/mpfit/mpfit.py}{\texttt{mpfit}}, you can do something like I did \href{https://github.com/laldoroty/LaurensTools/tree/main/LaurensTools/mpfit}{here}, which is copied in to Section \ref{sec:mpfit} (and you can \texttt{git clone} and then \texttt{pip install -e .} the entire \href{https://github.com/laldoroty/LaurensTools}{\texttt{LaurensTools}} package, if you want).

\subsubsection{Example: Fitting a line}
We have a model that we want to fit to our data,
\begin{equation}
    y(x) = mx + b,
\end{equation}
where $m$ is the slope and $b$ is the $y-$intercept. These are our parameters to be fit to our data. We want to find out what values of $m$ and $b$ fit the data best. So, our $\chi^{2}$ \textit{without} considering data error is:
\begin{equation}
    \chi^{2} = \sum_{i=0}^{k} (X_{i} - y(x_{i}))^{2},
\end{equation}
where $X_{i}$ is the $i$th data point, corresponding to independent variable $x_{i}$, and $y(x_{i})$ is the model at point $x_{i}$. Expanded, we have:

\begin{equation}
    \chi^{2} = \sum_{i=0}^{k} (X_{i} - (mx_{i} + b))^{2},
\end{equation}

If we want to consider error in both $x$ and $y$:
\begin{equation}
    \chi^{2} = \sum_{i=0}^{k} \frac{(X_{i} - (mx_{i} + b))^{2}}{\sigma_{y}^{2} + m^{2}\sigma_{x}^{2}}
\end{equation}


\subsubsection{Example: Fitting a Hubble diagram}

Let's say we want to use the following model for predicting distance using SNe Ia \cite{tripp1998}:

\begin{equation}
    \mu = m_{B} - M - \alpha(C - C_{avg})- \delta(\Delta m_{15} - \Delta m_{15, avg})
\end{equation}

The, the $\chi^{2}$ to minimize is
\begin{equation}
\begin{split}
    \chi^{2} = \sum_{i} \frac{(\mu_{i} - (m_{Bmax, i} - M -
    \alpha(C_{i}- C_{avg}) - \delta(\Delta m_{15, i} - \Delta m_{15,avg})))^{2}}{\sigma_{vpec, i}^{2} + \sigma_{Bmax, i}^{2} + (\alpha \sigma_{C, i})^{2} + (\delta \sigma_{\Delta m_{15}, i})^{2}}
    \end{split}
\end{equation}
$i$ represents each SN in the sample, and $\sigma_{vpec}$ is error in peculiar velocity, with $v_{pec} = 300 \mathrm{\,\,km\,\, s}^{-1}$.
\section{Observing}
Whelp, if you're reading this, I can only assume it's because you're about to irreparably mess up your week by staying up until heinous hours of the morning. Enjoy. 

\subsection{Making your observing plan}
\subsubsection{Exposure time}
I guarantee your telescope/instrument has an exposure time/integration time calculator. Go find it. 

\subsubsection{Signal-to-noise ratio}
The signal-to-noise ratio $(S/N)$ compares your signal to your noise. $(S/N) < 1$ means your noise is the predominant source, and $(S/N) > 1$ means your signal (or, the thing you actually \textit{want} to see in your observations) is the predominant source. It's defined like:

\begin{equation}
    S/N = \frac{S_{obj}}{S_{noise}}
\end{equation}

The higher your $(S/N)$, the better the observations. Note that your magnitude error $\sigma_{m}$ scales with $1/(S/N)$, i.e., higher $(S/N)$ means lower magnitude errors. Some key values to remember are: $(S/N) = 10 \implies \sigma_{m} \sim 0.1\times m$ and $(S/N) = 100 \implies \sigma_{m} \sim 0.01 \times m$. So... what does this mean? It means that $(S/N) \sim 10$ is probably the lowest you'll want to go for a given observation. If you need a quick spectrum, it's fine, you can do coarse measurements of large absorption lines here. You will most often encounter $(S/N)$ when you're deciding how long your exposure times need to be. \\

Unless you're building an exposure time calculator, you don't really need all the details on this equation, so I defer those detailed explanations to The Greater Internet$^{(\mathrm{TM})}$ because the point of this manual is to help you actually \textit{use} nebulous (lol) concepts.

% \subsubsection{Effective exposure time $t_{eff}$ (DES/DECam only)}
% $t_{eff}$ is the amount of time your exposure needs in perfect conditions. In other words,
% \begin{itemize}
%     \item Clear sky
%     \item Dark sky
%     \item PSF is the expected value
% \end{itemize}

% In general,

% \begin{equation}
%     t_{eff} = \frac{1}{\mathrm{FWHM}^{2}}10^{\Big( \frac{-2m_{cloud}}{2.5} - \frac{-m_{sky}}{2.5} \Big)}
% \end{equation}

\subsection{Observing logs}
If you're observing, you should keep a log that tracks what you did, when you did it, weather conditions, seeing, etc. If something goes wrong, write it down. This may be important down the line! Nothing is too unimportant to jot down. If you're in doubt, just write it down. 

\subsection{There's extra time! Or I ran out of time!}
\subsubsection{Extra time}
If your observing program provided extra targets, take a look at their RAs before you begin the night. Look at the schedule to see when you're observing objects with similar RA. If you're ahead of schedule while you're doing these, go ahead and add in the extra! If you don't know what objects to add, you can always repeat observations. No one will ever be mad about this. The most important thing is to check that the RA for the additional observation makes sense for the current time, and the easiest way to do this is to make sure it's close to the RA of the object(s) you most recently observed. 

\subsubsection{Ran out of time}
Nothing you can really do about this. If you've hit morning and you haven't observed everything, that's that. If it's the middle of the night and you're behind schedule, you can make some decisions to skip targets. If you know the weather is going to be bad, it doesn't hurt to find out which objects are lowest priority. Always make a note in the observing log if you chose to skip something, and write down why. It's probably not your fault---the schedule may have been a bit off, or weather forced you to close the dome. No one will criticize you for this! These things are out of your control. 

\subsection{Remote}
\subsubsection{Before you begin the night...}
There are a few things you need to do before you begin the night. First, make sure you have all the necessary software installed and you know how to access all in-browser GUIs. You might need:
\begin{itemize}
    \item A VPN client. I use Cisco AnyConnect (at the time of writing this, Texas A\&M provided this software, so any instructional VPN-related screenshots will be using Cisco AnyConnect on an Ubuntu system). If you're an A\&M affiliate, you can get it through \href{connect.tamu.edu}{connect.tamu.edu}. You \textit{must} have Duo set up to do this. I spent like, two weeks trying to figure out why I couldn't get the software until I realized it was because I wasn't getting push notifications from Duo on my phone. 
    \item A VNC viewer. \href{https://www.realvnc.com/en/connect/download/viewer/}{Real VNC} is nice and it is free. 
    \item A stable internet connection. If you're doubtful of your home internet, cut your losses and go to your office. Logging back in to everything every time you lose connection will be extremely annoying. 
    \item Something(s) to keep yourself awake. Food, caffeine, music, games, etc. Observing is your priority though, so whatever it is, make sure you can drop it if you need to. For example, I do not recommend League of Legends, but I do recommend \href{https://www.stardewvalley.net/}{Stardew Valley}. A lot of people get work done during the night, but personally, I am incapable of being useful past 10:00 PM. 
\end{itemize}
\par
Second, it helps to organize your scripts/plan in advance. Do whatever you want, but I like to put scripts in separate folders for $\sim1$ hr blocks of time so I don't have to think about where/when all my targets are at night when my brain is running on steam. 

\subsubsection{CTIO/DECam}
\par
You should always know what the local time is. For CTIO, the time is the same as the \href{https://time.is/Santiago}{time in Santiago, Chile}. \\

In general, in this section, usernames/passwords/addresses will be obscured or not provided. You should get these from support. If you need your proposal ID, check the \href{https://noirlab.edu/science/observing-noirlab/scheduling/mso-telescopes}{schedule for the 4m telescope at CTIO}. 

First, make sure you have installed:
\begin{itemize}
    \item A VPN client
    \item A VNC viewer
\end{itemize}

You can join the Zoom call any time between now and when you need to request observer permissions. If you have trouble with something, join the Zoom call and ask for help! \\

The first thing you need to open in your computer is your VPN. You need to do this to access the network at CTIO. Log in like in Figure \ref{fig:vpn}, but get your address, group, username, and password from support. \\

\begin{figure}[h!]
    \centering
    \includegraphics[width=0.6\textwidth]{figs/observing/vpn.png}
    \caption{This is what your VPN window should look like for DECam, if you're using Cisco AnyConnect.}
    \label{fig:vpn}
\end{figure}

Open up your VNC viewer. Enter the address and password. RealVNC will save this information for next time. The VNC viewer allows you to remotely use a computer at CTIO. \\

I think it's also nice to open the \href{http://www.ctio.noao.edu/noao/content/ctio-external-webcam}{CTIO external webcams}. It's the only view of the sky at CTIO you'll get all night. Plus, it's a quick way to check the weather. A slightly less quick, but more quantitative way to check the weather is to use the \href{https://noirlab.edu/science/observing-noirlab/weather-webcams/cerro-tololo/environmental-conditions}{CTIO Site Environmental Conditions page}.\\
\par
Now, open the \href{http://system1.ctio.noao.edu:7001/apps/}{SISPI GUIs}. You'll see a bunch of options, like in Figure \ref{fig:sispi}. If you're not on the VPN, you won't be able to access this page. Open one of apps in a new tab and enter the username and password provided by support. If it's not working and you've copied and pasted the login info in to the text boxes, try typing it in by hand. This has been an issue for me in the past, I'm unsure if it's inherent to the software, or if it's a browser issue, or something else. \\ 

\begin{figure}
    \centering
    \includegraphics[width=0.5\textwidth]{figs/observing/sispi.png}
    \caption{The SISPI GUIs, with the important ones (or so I think) highlighted in green.}
    \label{fig:sispi}
\end{figure}

In the Observer Console app, which you access from the SISPI GUI Interfaces page, the first thing you need to do is request observer permissions. In the upper right-hand corner, there's a padlock button (see the left panel of Figure \ref{fig:request}). Click this. Then, a window will pop up like in the right panel of Figure \ref{fig:request}. Enter your proposal ID, change the ``level'' button from ``user'' to ``observer'', and click ``confirm''. The username will already be entered. Kindly ask your support staff on Zoom to grant you observer permissions. \\

\begin{figure}
    \centering
    \includegraphics[width=0.46\textwidth]{figs/observing/request.png}
    \includegraphics[width=0.49\textwidth]{figs/observing/request_2.png}
    \caption{\textit{Left}: The location to click to begin requesting observer permissions. \textit{Right}: The pop-up window to request observer permissions.}
    \label{fig:request}
\end{figure}

Once you're in, there are a few things you can do. First, update the Observers, Proposal ID, Program, and Investigator information if it hasn't already been updated. You'll find these in the ``System Control'' tab along the top of the screen. Click ``Edit'', edit the text boxes, and then click ``Save''. In the ``Exposure Control'' tab, you upload the actual json scripts. Click ``Load exposure script'', ``Browse...'' and then ``Submit'' (this may appear as ``Sub...''). It'll appear in the exposure queue as a block of yellow, green, or blue exposures. The colors don't mean anything, they're just to differentiate between which exposures belong to which scripts. \\

\begin{figure}
    \centering
    \includegraphics[width=0.5\textwidth]{figs/observing/unlock.png}
    \caption{The exposure queue, with the padlock icon to allow deleting/shuffling of exposures as-needed. Target information from this screenshot is greyed out. To the left of the padlock, ``45'' indicates that there are 45 exposures currently loaded into the queue. Underneath these icons, the ``Stop'' button (which is often red, not yellow, I don't know why it's yellow here) will interrupt the exposure loop. When pressed, it will change to ``Go''. This flip-flops depending on if the exposure loop is running.}
    \label{fig:unlock}
\end{figure}

Once your stuff is in the exposure queue, you can move around or delete the exposures as needed. Click the padlock icon, highlighted in Figure \ref{fig:unlock}, to unfreeze the queue. Now, you can delete things or move them around if you need to. This is uncommon, but it happens. When you're done, \textit{always re-lock it}. To the left of the padlock, there are numbers that switch between integers and a timer. The timer is the amount of exposure time remaining in the queue. The integer is the number of exposures remaining in the queue. \\

In the Telemetry Viewer app, I like to look at the Pointing and Image Health tabs. All plots indicate the values calculated for the \textit{previous} image. They are not updated in real-time. The Pointing tab has pointing and center offsets plots---you'll want to keep an eye on these to make sure these don't drift too far from center. Image Health has a seeing plot, which shows the... seeing. Write this down in your observing log every so often. You don't have to be super precise. This number gets recorded in the FITS header anyway. \\

The Comfort Display SISPI GUI is also important. You should look at all exposures here, in addition to some in the VNC viewer to check for image saturation. If your background counts are $>40,000$, the detector is definitely saturated. \textbf{You must stop saturated observations immediately}. They can damage the instrument. $>20,000$ is maybe not instrument-damaging, but they're a waste of time. Expect the $g-$band to be saturated during bright time, and the $r-$band if you're pointing close to the Moon or have a very long exposure ($>100$s) in bright time. You CAN stop an exposure in the middle and look at counts before continuing if you're concerned it may saturate the detector. \\

You'll want to keep track of the Alarm History GUI as well; if there's a warning, you should figure out what's causing it. Finally, open the Guider app. This will tell you if there are guiding issues, which can arise if you're in a crowded field. \\

% SISPI GUIs that are important: Comfort display (observer should ideally look at ALL exposures that are taken here, plus some on the VNC viewer for e.g. image saturation during bright time), alarm history (you need to know of all warnings and alarms and investigate what's causeing them), guider (so you can see if there are guiding issues e.g. in crowded fields, and keep an eye on ellipticity)

Let's talk about the center offset correction. In the VNC viewer, if a terminal is not already open, open a terminal. Type \texttt{observer}, hit enter, then \texttt{center}, then hit enter. This will print the center offset for the previous image. If it's large, kindly ask your support staff in the Zoom call for a pointing correction. Listen to their instructions about hitting ``Stop'' and ``Go'' in the exposure queue (the correction can't be made in the middle of an exposure---you need to interrupt the exposure loop). I usually ask for a correction around a 10'' offset. If you're near the end of observing an object, don't worry about it, wait until the telescope slews to see if it corrects itself. If not, go ahead and ask.\\

Now that I've said all that, I'll go over using the terminal in the VNC viewer. You'll use a package called Kentools, which you enter in the terminal by typing ``observer''. Exit by typing ``exit''. Like this: 

\begin{minted}[
    bgcolor=lightgray,
    frame=leftline,
    framesep=-3mm]{c}

    # Open Kentools:
    observer2> observer
    # Close Kentools:
    prompt> exit
\end{minted}

That's it! It'll pull up a long list of commands, which I will not copy and paste here. To check centering and seeing, do this: 

\begin{minted}[
    bgcolor=lightgray,
    frame=leftline,
    framesep=-3mm]{c}

    # Open Kentools:
    observer2> observer
    # List commands:
    prompt> commands
    # Check centering for previous image:
    prompt> center
    # Check seeing for the previous image:
    prompt> seeingall
\end{minted}

You can open the images you're taking and look at them in DS9! This is also how you check counts. 

\begin{minted}[
    bgcolor=lightgray,
    frame=leftline,
    framesep=-3mm]{c}

    # From Kentools, check image inventory
    prompt> inv
    # Load the S4 CCD for the previous image (default). 
    prompt> load
\end{minted}

DS9 will pop up with the image. You can check the background counts by hovering your mouse over any location in the image and reading what's in the ``value'' box underneath the ``File'' and ``Object'' information fields.  Here's another example:
\begin{minted}[
    bgcolor=lightgray,
    frame=leftline,
    framesep=-3mm]{c}

    # From Kentools, check image inventory
    prompt> inv
    # I picked exposure number 1156247, and load the N5 CCD. 
    prompt> load 1156247 36
\end{minted}

\begin{figure}[h!]
    \centering\includegraphics[width=0.8\textwidth]{figs/observing/DECamOrientation.png}
    \caption{Alphanumeric codes for each DECam CCD.}
    \label{fig:decamccds}
\end{figure}

You can get the exposure number from several places. You can type ``inv'' in Kentools to get an inventory of the exposures from that night. The first column is the exposure number. You can also get the most recent exposure number from the dialogue box in the exposure console, or the upper left-hand corner of the comfort display GUI. You can get the CCD code from Figure~\ref{fig:decamccds}. The N5 CCD has a green 36 written next to it, so I type ``36'' to specify that I want to look at that particular CCD. You can also look at the ENTIRE frame with ``bigload''. Same deal as ``load'', but you don't need to specify the CCD because it loads all the CCDs: \texttt{bigload [expnum]}. \\

At the \href{http://www.ctio.noao.edu/noao/content/End-Night-2}{end of the night}, you need to fill out the \href{http://www.ctio.noao.edu/noao/node/add/night-report}{end-of-night report}. If you observed during the first half of the night, you need to create it. If you observed during the second half, you will get the link from the first half observer (unless they didn't start it. Then you need to make it). Get the login credentials from support. It is very important to ask the telescope operator who needs to be reported in the night log! Don't be afraid to ask.\\

At the end of the night, grab the inventory file. In the terminal in your VNC viewer (or if you SSH'd in via a terminal on your computer):

\begin{minted}[
    bgcolor=lightgray,
    frame=leftline,
    framesep=-3mm]{c}

    # From Kentools:
    prompt> godb
    prompt> invPrint
    prompt> qcInvPrint
\end{minted}

Then, on your local computer, i.e., in a terminal \textit{without} the SSH connection, 
\begin{minted}[
    bgcolor=lightgray,
    frame=leftline,
    framesep=-3mm]{c}

    scp DECamObserver@observer2.ctio.noao.edu:/user/DECamObserver/
        yyymmdd.qcinv .
\end{minted}

where \texttt{yyymmdd} is year, month, day. Also, don't drop the period at the end of this command! You'll need to put in the SSH password, and then a file called \texttt{yyyymmdd.qcinv} will appear in the folder from where you ran the \texttt{scp} command.
\section*{Acknowledgements}
The author thanks the following people for useful comments: Peter Brown, D'Arcy Kenworthy, Antonella Palmese, Nick Suntzeff, and Jiawen Yang.
\bibliography{refs}{}
% \bibliographystyle{aasjournal}
\bibliographystyle{unsrt}

\end{document}
