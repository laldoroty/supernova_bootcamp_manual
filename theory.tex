\section{Theory}
Haha, we weren't going to get away without a theory section. Haha. hahahahahaha.

\subsection{Deflagration vs. Detonation}

\subsection{Arnett's rule}

\subsection{Important publications}

\textit{Early Supernova Luminosity}, Colgate and McKee 1969 \cite{colgatemckee1969}. First paper that shows that $^{56}$Ni is a ``predominant end product of thermonuclear processes'', and thus may be synthesized in large quantities in the ejecta of SNe Ia (but they were just called ``Type I'' back then). Astronomy is introduced to the decay sequence of $^{56}$Ni: $^{56}$Ni $\rightarrow$ $^{56}$Co $\rightarrow$ $^{56}$Fe. $^{56}$Co decay is mentioned, but no conclusions about it are drawn.\\

\textit{On the Theory of Type I Supernovae}, Arnett 1979 \cite{arnett1979}. Uses the idea from Colgate and McKee 1969 \cite{colgatemckee1969} to show that, actually, yeah, $^{56}$Ni decay is a totally reasonable explanation for Type I supernova light curves. This decay entirely explains the bolometric light curve near peak brightness, and at later times ($t \gtrsim 100$ d), the exponential nature of the light curve can be explained by $^{56}$Co decay. \\

\textit{Type I supernovae. I - Analytic solutions for the early part of the light curve}, Arnett 1982 \cite{Arnett1982}. \\

\textit{Seeing the Collision of a Supernova with its Companion Star}, Kasen 2010 \cite{kasen2010}.